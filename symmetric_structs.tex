\section {Relational Colour Refinement for symmetric structures}

One very interesting subclass of relational structures is the class of symmetric structures.
A special case of these are for example undirected graphs, as their edge relations are symmetric.
This notion of symmetry can be generalized to any relational signature.

\begin{definition}[Symmetric Structures]
	Let $\sigma$ be a relational signature.
	A structure $\mathfrak A$ of signature $\sigma$ is a symmetric structure, if for every relation and every tuple in those relations, the order of the elements is irrelevant.
	This means, that every relation $R$ with arity $k$ is a subset of all possible subsets of $A$ with exactly $k$ elements.
	Formally, we have
	$$R\subseteq \binom{A}{k}.$$
\end{definition}
An equivalent characterisation uses the symmetric groups $\mathcal S_k$.
We call a $\sigma$ structure $\mathfrak A$ symmetric, if for every $R\in \sigma$ of arity $k$, every $k$-tuple $\mathbf x=(x_1,x_2,\dots,x_k)\in R^{\mathfrak A}$ and every $k$-permutation $\pi\in \mathcal S_k$, we have that
$$(x_{\pi(1)},x_{\pi(2)},\dots,x_{\pi(k)})\in R^{\mathfrak A}.$$
In the following, we will use $\pi(\mathbf x)$ as a shorthand notation for $(x_{\pi(1)},x_{\pi(2)},\dots,x_{\pi(k)})$.

As symmetric structures are a subset of relational structures, the results from \cite{scheidt2025ColorRefinement} obviously apply to them.
Thus, we have that the following three statements are equivalent for two symmetric $\sigma$ structures $\mathfrak A$ and $\mathfrak B$:
\begin{enumerate}
	\item $\RCR$ distinguishes $\mathfrak A$ and $\mathfrak B$.
	\item There exists a sentence $\phi\in\GFC$, such that $\mathfrak A\models \phi$ and $\mathfrak B\not\models \phi$.
	\item There exists an acyclic $\sigma$ structure $\mathfrak C$, such that $\hom(\mathfrak C,\mathfrak A)\neq\hom(\mathfrak C,\mathfrak B)$.
\end{enumerate}
However, as we restricted the class of structures for $\mathfrak A$ and $\mathfrak B$, this poses the question, whether the same can be done to the acyclic structures.
Concretely, we want to investigate, whether the first statement is also equivalent to there being an acyclic, symmetric $\sigma$ structure, such that it has a different homomorphism count to $\mathfrak A$ than to $\mathfrak B$.

As we will prove in the following, it is indeed the case that we can restrict the class of acyclic structures to only include structures that acyclic and symmetric.
However, before we prove this, we have to show a lemma which will be used in the proof.
As a reminder on notation, for a $k$-tuple $\mathbf x=(x_1,x_2,\dots,x_k)$ a homomorphism $\phi$ and a permutation $\pi$ we write $\phi(\mathbf x)$ for $(\phi(x_1),\phi(x_2),\dots,\phi(x_k))$ and $\pi(\mathbf x)$ for $(x_{\pi(1)},x_{\pi(2)},\dots,x_{\pi(k)})$.
\begin{lemma}
	Let $\pi\in\mathcal S_k$, $\phi$ be a homomorphism, $R$ a $k$-ary relation and $\mathbf x\in R$.
	Then $\phi(\pi(\mathbf x))=\pi(\phi(\mathbf x))$.
\end{lemma}
\begin{proof}
	We prove this by contradiction.
	Assume the contrary.
	Then there exists an $i\in[k]$, such that $\phi(\pi(\mathbf x))_i\neq \pi(\phi(\mathbf x))_i$.
	Note that the definitions of $\phi(\pi(\mathbf x))$ and $\pi(\phi(\mathbf x))$ are 
	$$\phi(\pi(\mathbf x)) = (\phi(x_{\pi(1)}), \phi(x_{\pi(2)}),\dots,\phi(x_{\pi(k)}))$$
	and
	$$\pi(\phi(\mathbf x)) = (\phi(\mathbf x)_{\pi(1)},\phi(\mathbf x)_{\pi(2)},\dots,\phi(\mathbf x)_{\pi(k)}).$$
	From these, we directly get
	$$\phi(\pi(\mathbf x))_i=\phi(x_{\pi(i)})=(\phi(x_1),\phi(x_2),\dots,\phi(x_2))_{\pi(i)}=\phi(\mathbf x)_{\pi(i)}=\pi(\phi(\mathbf x))_i.$$
	Contradiction!
	Therefore the lemma must hold.
\end{proof}
We now prove the above claim:
\begin{theorem}
	Let $\sigma$ be a relational signature and $\mathfrak A$ and $\mathfrak B$ be two $\sigma$ structures.
	Then the following two statements are equivalent:
	\begin{enumerate}
		\item $\RCR$ distinguishes $\mathfrak A$ and $\mathfrak B$.
		\item There exists an acyclic, symmetric $\sigma$ structure $\mathfrak C$ with $\hom(\mathfrak C,\mathfrak A)\neq \hom(\mathfrak C,\mathfrak B)$.
	\end{enumerate}
\end{theorem}
\begin{proof}
	We first prove that \textit{2.} implies \text{1.}.
	Let $\mathfrak C$ be a acyclic, symmetric $\sigma$ structure with $\hom(\mathfrak C,\mathfrak A)\neq\hom(\mathfrak C,\mathfrak B)$.
	As $\mathfrak C$ is acyclic, we can apply \ref{} and get that $\RCR$ must distinguish $\mathfrak A$ and $\mathfrak B$.
	
	We now prove that \textit{1.} implies \textit{2.}.
	Assume that $\RCR$ distinguishes $\mathfrak A$ and $\mathfrak B$.
	From \ref{} we know that there exists an acyclic structure $\mathfrak C'$ with $\hom(\mathfrak C',\mathfrak A)\neq \hom(\mathfrak C',\mathfrak B)$.
\end{proof}



