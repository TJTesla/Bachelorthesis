\section{Introduction}

The graph isomorphism problem is a very interesting and important problem in both theoretical and applied computer science \cite{grohe2020GraphIsomorphism}.
Its time-complexity has been studied intensely and many different algorithms and approaches have been devised.
One algorithm that can be used for isomorphism testing is \emph{Colour Refinement}, short $\CR$, also called the \emph{1-dimensional Weisfeiler-Leman algorithm}.
Given two graphs, it can prove in quasilinear time that they are not isomorphic \cite{berkholz2017TightLowera}.
Concretely, $\CR$ is an iterative algorithm, that, in the beginning, assigns every vertex the same colour and in following iterations assigns each one a new colour, based on the colours of its neighbours.
This procedure get repeated, until the partition of the vertices induced by the colouring stays the same. 
% That colouring is then called the \emph{stable colouring}.
We then say that Colour Refinement distinguishes two graphs, if there is some colour that appears differently often in the two graphs.
It is easy to see that two isomorphic graphs are not distinguished by Colour Refinement.
This is equivalent to the fact that if two graphs get distinguished by Colour Refinement, they cannot be isomorphic.
Furthermore, while it is not possible to infer the opposite direction, that is two non-isomorphic graphs always get distinguished by $\CR$, it was shown by Babai, Erd\H{o}s and Selkow that almost all graphs get distinguished by it \cite{babai1980RandomGraph}.
However, there exist some classes of graphs, that cannot be distinguished by Colour Refinement, for example the class of regular graphs with the same number of vertices.

One of the most interesting aspects of Colour Refinement, is its characterisability through homomorphism counting and logic.
Given a tree $T$ and two graphs $G$ and $H$, we define $\hom(T, G)$ and $\hom(T,H)$ as the number of homomorphisms from $T$ to $G$ and $H$, respectively.
Then, by the results of Dvo\v r\'ak \cite{dvorak2010RecognizingGraphsa} and Dell, Grohe and Ratten \cite{dell2018LovaszMeets} we have $\hom(T, G)\neq\hom(T, H)$ if, and only if, Colour Refinement distinguishes $G$ and $H$.
Such a characterisation can also be done through logic.
We define $\C{2}$ as the logic that extends first-order logic by counting quantifiers of the form $\exists^{\geq i} x.\phi(x)$ with the semantic that there must be two distinct values $x_1$ and $x_2$ such that $\phi(x_1)$ and $\phi(x_2)$ are fulfilled.
For a sentence $\phi\in \C{2}$ and two graphs $G$ and $H$, it was the shown by Cai and Immerman \cite{cai1992OptimalLower} and Immerman and Lander \cite{immerman1990DescribingGraphs} that $G\models \varphi \Leftrightarrow H\not\models \varphi$ if, and only if, Colour Refinement distinguishes $G$ and $H$.

Aside from isomorphism testing, Colour Refinement has applications in different fields.
Incidentally, the first recorded occurrence of this algorithm appeared in 1965 and dealt with the description of chemical structures \cite{morgan1965GenerationUnique}.
Its significance for computer science has been recognised later by Weisfeiler and Leman in 1968 \cite{weisfeiler1968reduction}.
One interesting application of Colour Refinement is in the reduction of the dimension of linear programs.
By defining a variant of Colour Refinement on matrices which finds a partition of the rows and columns, it is possible to reformulate a linear program with a considerably smaller dimension. 
This method of first reducing the problem and then solving the reduced instance has been shown to be more performant than the standard way of solving linear programs. \cite{grohe2014DimensionReduction}
Another application can be found in the field of machine learning, more precisely for kernel methods.
The aim of this method is to assign a similarity value between two elements, which then can be used in more complex machine learning techniques such as support vector machines or regression.
An emerging concept in this field are kernels for graphs, also called graph kernels, which are a method to compare two graphs, and represent how similar they are with a single value.
The usual method of classical graph kernels is to consider certain subgraphs for the calculation.
A further analysis of this method can be found in \cite{vishwanathan2010graph}.
One interesting application of Colour Refinement can be found in its usage as a graph kernel.
When fixating an integer $h$ and counting for each of the first $h$ Colour Refinement steps how many vertices between two graphs share the same colour, we get a graph kernel.
This Weisfeiler-Leman Graph Kernel has an adequate ability to to classify graphs, while having a significantly better runtime than classical graph kernels. \cite{grohe2021ColorRefinement}

From the above examples it can be seen that while Colour Refinement is a simple procedure, it can be applied in various situations.
This versatility has been one of the reasons for its success and poses the question, whether it could be possible to formulate an analogous procedure for more than graphs.

% Relational structures: 2021, CR on incidence graph, Berge acyclicity
% Hypergraphs: 2019, Only hom counting, 

% CR on Relational structures (RCR)

% My results

