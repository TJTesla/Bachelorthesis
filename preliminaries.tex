\section{Preliminaries}
\label{sec:Preliminaries}

% Tuple notation
Let $A$ be a set.
Then any $a_1,\dots, a_k\in A$ form a tuple with arity $k$ of the form $\mathbf a=(a_1,\dots,a_k)$ and with $\mathbf a_n$ we denote $a_n$.
For any function $f:A\to B$, we write $f(\mathbf a)$ for $(f(a_1),\dots,f(a_k))$.
As a special case, if $\pi$ is a permutation, that is $\pi\in \mathcal S_k$, where $\mathcal S_k$ is the symmetric group for $k$ elements, then $\pi(\mathbf a)$ means $(a_{\pi(1)},\dots,a_{\pi(k)})$.
For a set $A$ and a $k\in \mathbb N$, we write $\binom{A}{k}$, to denote all subsets of $A$ with cardinality exactly $k$.
Usually, sets are denoted by uppercase, elements from sets as lowercase, and tuples as boldface and lowercase Latin letters.

% Natural numbers
By $\mathbb N$ we mean the set of natural numbers, including $0$, and with $\mathbb N_{\geq k}$ for a $k\in \mathbb N$, we denote the set $\{k,k+1,\dots\}=\mathbb N \setminus \{0,\dots,k\}$.

% Multisets
A multiset $\mathcal M$ is a set $A$ with a function $\operatorname{mult}_{\mathcal M}:A\to \mathbb N_{\geq 1}$.
This denotes, that for every $a\in A$, there are exactly $\operatorname{mult}_{\mathcal M}(a)$ many copies.
We also write multisets as $\multiset{\dots}$.
As an example, the multiset $\mathcal M = \multiset{a, b, b}$ is equivalent to the set $A=\{a,b\}$, together with the function $\operatorname{mult}_{\mathcal M}=\{a\mapsto 1, b\mapsto 2\}$.

% Signatures and Structures and Graphs
A signature is a set of relation and function symbols, paired with an arity for every symbol.
We write relation-symbols as uppercase and function-symbols as lower case letters.
For example, $\sigma=\{R/2, T/3, f/3\}$ represents a signature with a relation-symbol $R$ with arity $R$, a relation-symbol $T$ with arity $3$ and a function-symbol $f$ with arity $3$.
For a signature $\sigma$ we have that a $\sigma$-structure $\mathfrak A$ is a tuple $(A,\sigma)$ where $A$, called the universe, is a set of elements and for every $k$-ary relation-symbol $R\in \sigma$ and every $\ell$-ary function-symbol $f\in \sigma$, there is a relation $R^{\mathfrak A}\subseteq A^k$ and a function $f^{\mathfrak A}:A^\ell \to A$.
We define $\vert \mathfrak A\vert =\vert A \vert$.
A structure is always written as an uppercase Fraktur letter, its universe is written using the same uppercase normal letter, and the letter written boldface denotes the set of all tuples.
As an example, $\mathfrak A$ is a structure, $A$ is its universe and $\mathbf A$ is the set of its tuples, formally $\{\mathbf a : \mathbf a \in R^{\mathfrak A} \text{ for a } R\in\sigma\}$.
A graph $G$ has a set of vertices, called 
If not specified, we only talk about finite signatures and finite universes.

% Colour Refinement (Basic logic (C^2), Hom/hom)