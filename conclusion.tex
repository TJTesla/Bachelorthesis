\section{Conclusion}

In this thesis, we presented the results of Scheidt and Schweikardt \cite{scheidt2025ColorRefinement} and discussed how they can be extended for the use with non-relational structures.
We presented two possible ways on how non-relational structures can be encoded as relational structures and investigated the logical and combinatorial characterisations.
Naive Relational Colour Refinement, where a function is directly interpreted as a relation, is characterised by the logic $\mathsf{nfGF}(\mathsf C)$, which uses function symbols like relation symbols.
By using the transitive expansion of a function as its encoding, we can allow arbitrarily many function applications, with a bound on the number of alternations. 
This notion is captured by the logic $\GFC_k$.
However, while we find a logical characterisation, the combinatorial characterisation is not possible.
We showed that the existence of an acyclic, total, functional relational structure that distinguishes two encoded structures by homomorphism count is equivalent to the existence of a non-relational structure that also distinguishes the structures.
But while it is possible to construct a functional structure, it is not possible to enforce totality.
This then lead us to the question of investigating restrictions of the class of structures and for which restrictions the characterisation by homomorphism counting remains.
Aside from the negative result in that regard for total structures, we showed that it is possible to restrict the class to symmetric structures.

There are multiple questions that are still unanswered.
\begin{itemize}
	\item When characterising $\RCR_k$ logically, we were able to show that the number of applications of a single function symbol does not need a bound.
	The same was not done for the alternation depth.
	One interesting question would be to investigate, whether it is possible to only consider the transitive expansion up to a certain alternation depth $d$.
	Furthermore, would it then be possible to allow any alternation depth in a formula to characterise this algorithm?
	The result would then be that $\GFC$ with the standard definition atomic formulae would characterise $\RCR$ over non-relational structures.
	\item Our two algorithms operate on different classes of structures.
	While nRCR is defined for any structure, $\RCR_k$ only works for structures over signatures with relation and unary function symbols.
	It is not clear, whether the used approach can be adapted for functions with arity $\geq 2$.
	However, an iterative colouring algorithm that is stronger than nRCR but still works on any structure is desirable.
	\item We have considered two possible restrictions on the class of relational structures for characterisation by homomorphism counting.
	This poses the question, which other restriction could be characterised this way.
\end{itemize}

