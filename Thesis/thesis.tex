%%%%%%%%%%%%%%%%%%%%%%%%%%%%%%%%%%%%%%%%%%%%%%%%%%%%%%%%%%%%%%%%%%%%%%%%%%%%%%%
% i7 Seminar Report Template
% Version November 8, 2023

\documentclass[a4paper,11pt,DIV=15]{scrartcl} % Do not edit this line.


%%%%%%%%%%%%%%%%%%%%%%%%%%%%%%%%%%%%%%%%%%%%%%%%%%%%%%%%%%%%%%%%%%%%%%%%%%%%%%%
% Preamble

% Page Geometry, Typography and Encoding
\usepackage[utf8]{inputenc}
\usepackage[T1]{fontenc}
\usepackage{microtype}
\usepackage{lmodern}
\renewcommand{\phi}{\varphi}
\renewcommand{\epsilon}{\varepsilon}
% \renewcommand{theta}{\vartheta} % if you want

% Math packages
\usepackage{amsmath}
\usepackage{amssymb}
\usepackage{amsthm}
\usepackage{mathtools}

% Floats
\usepackage{float}
\usepackage{booktabs}
\usepackage{tikz}
\usetikzlibrary{positioning,arrows.meta}

% Colors
\usepackage{xcolor} %already loaded by tikz, but here for completeness
% RWTH colors
% blue violet purple carmine red magenta orange yellow grass cyan gold silver
\definecolor{rwth-blue}{cmyk}{1,.5,0,0}\colorlet{rwth-lblue}{rwth-blue!50}\colorlet{rwth-llblue}{rwth-blue!25}
\definecolor{rwth-violet}{cmyk}{.6,.6,0,0}\colorlet{rwth-lviolet}{rwth-violet!50}\colorlet{rwth-llviolet}{rwth-violet!25}
\definecolor{rwth-purple}{cmyk}{.7,1,.35,.15}\colorlet{rwth-lpurple}{rwth-purple!50}\colorlet{rwth-llpurple}{rwth-purple!25}
\definecolor{rwth-carmine}{cmyk}{.25,1,.7,.2}\colorlet{rwth-lcarmine}{rwth-carmine!50}\colorlet{rwth-llcarmine}{rwth-carmine!25}
\definecolor{rwth-red}{cmyk}{.15,1,1,0}\colorlet{rwth-lred}{rwth-red!50}\colorlet{rwth-llred}{rwth-red!25}
\definecolor{rwth-magenta}{cmyk}{0,1,.25,0}\colorlet{rwth-lmagenta}{rwth-magenta!50}\colorlet{rwth-llmagenta}{rwth-magenta!25}
\definecolor{rwth-orange}{cmyk}{0,.4,1,0}\colorlet{rwth-lorange}{rwth-orange!50}\colorlet{rwth-llorange}{rwth-orange!25}
\definecolor{rwth-yellow}{cmyk}{0,0,1,0}\colorlet{rwth-lyellow}{rwth-yellow!50}\colorlet{rwth-llyellow}{rwth-yellow!25}
\definecolor{rwth-grass}{cmyk}{.35,0,1,0}\colorlet{rwth-lgrass}{rwth-grass!50}\colorlet{rwth-llgrass}{rwth-grass!25}
\definecolor{rwth-green}{cmyk}{.7,0,1,0}\colorlet{rwth-lgreen}{rwth-green!50}\colorlet{rwth-llgreen}{rwth-green!25}
\definecolor{rwth-cyan}{cmyk}{1,0,.4,0}\colorlet{rwth-lcyan}{rwth-cyan!50}\colorlet{rwth-llcyan}{rwth-cyan!25}
\definecolor{rwth-teal}{cmyk}{1,.3,.5,.3}\colorlet{rwth-lteal}{rwth-teal!50}\colorlet{rwth-llteal}{rwth-teal!25}
\definecolor{rwth-gold}{cmyk}{.35,.46,.7,.35}
\definecolor{rwth-silver}{cmyk}{.39,.31,.32,.14}

% Hyperlinks and Cross-References
\usepackage{hyperref}
\usepackage[capitalise,noabbrev]{cleveref}
\hypersetup{%
	pdftoolbar=false,
	pdfmenubar=false,
	colorlinks,
	%pdfborderstyle={/S/U/W 1.25},
	urlcolor={rwth-magenta},
	linkcolor={rwth-red},
	citecolor={rwth-green}
}

\theoremstyle{plain}
\newtheorem{theorem}{Theorem}
\newtheorem{proposition}[theorem]{Proposition}
\newtheorem{lemma}[theorem]{Lemma}
\newtheorem{corollary}[theorem]{Corollary}
\newtheorem{conjecture}[theorem]{Conjecture}
\newtheorem{claim}[theorem]{Claim}
\theoremstyle{definition}
\newtheorem{definition}[theorem]{Definition}
\newtheorem{remark}[theorem]{Remark}



% Misc packages
\usepackage{lipsum}


% Custom commands
\newcommand{\GFC}{\mathsf{GF}(\mathsf{C})}
\newcommand{\free}[1]{\operatorname{free}(#1)}
\newcommand{\gd}[1]{\operatorname{gd}(#1)}

\renewcommand{\theta}{\vartheta}


%%%%%%%%%%%%%%%%%%%%%%%%%%%%%%%%%%%%%%%%%%%%%%%%%%%%%%%%%%%%%%%%%%%%%%%%%%%%%%%
% Document


\begin{document}

%TODO Insert topic of seminar, e.g. Theoretical Topics in Data Science or Complexity Theory
\subtitle{Bachelor's Thesis}
\date{August 28, 2025}
\publishers{RWTH Aachen University}	% Do not edit this line.

%TODO Change this to your report title.
\title{Relational Colour Refinement for Non-Relational Signatures}

%TODO Change this to your name.
\author{Theodor Teslia}

\maketitle


%TODO Provide a short abstract for your report.
\begin{abstract}
	\lipsum[1-2]
\end{abstract}

\thispagestyle{empty}

\clearpage

%TODO The content of your report goes below.


\section{Introduction}

\lipsum[2-3]

\section{Relational Colour Refinement}

\lipsum[4-5]

\clearpage

\section{Relational Colour Refinement for structures with functions}

\subsection{Naive Encoding of functions}

A simple way to apply relational colour refinement to non-relational structures is, to encode the functions in the signature as a relation.
Formally we transform a signature $\sigma$ that includes function symbols to a new signature $\sigma'$: 
For every relation symbol $R\in \sigma$, we introduce a relation symbol $R\in \sigma'$ with the same arity and for every function symbol $f\in\sigma$ with arity $k$, we introduce a relational symbol $R_f\in\sigma'$ of arity $k+1$.

Semantically, a structure $\mathfrak A$ of signature $\sigma$ can then be encoded as a structure $\mathfrak A'$ of signature $\sigma'$ and with the same universe as $\mathfrak A$. 
For every relational symbol $R\in\sigma$ we set $R^{\mathfrak A'}\coloneqq R^{\mathfrak A}$ and for every function symbol $f\in\sigma$ of arity $k$ there exists a relation symbol $R_f\in\sigma'$ and we set $R_f^{\mathfrak A}\coloneqq \{(\mathbf x, y) : f^{\mathfrak A}(\mathbf x)=y\}$ where $\mathbf x$ is a tuple of arity $k$.

This procedure encodes a non-relational structure as a relational one, on which Relational Colour Refinement can now be performed.
As such we say, that the Naive Relational Colour Refinement (nRCR) distinguishes two structures $\mathfrak A$ and $\mathfrak B$ if, and only if, RCR distinguishes their naive encodings $\mathfrak A'$ and $\mathfrak B'$.
However, this results in a very weak logical characterisation, that does not allow nesting of terms, namely the nesting-free-fragment of $\GFC$.

\begin{definition}[$\mathsf{nfGF}(\mathsf C)$]
	Consider the definition of $\GFC$ given in \ref{}.
	We obtain the nesting-free fragment, by allowing $f(\mathbf x)=y$ as a further atomic formula.
	Concretely, the only allowed atomic formulae are of the form $R(x_1,\dots,x_\ell)$, $x=y$ and $f(x_1,\dots,x_\ell)=y$, where $f$ has arity $\ell$, $\free{f(x_1,\dots,x_\ell)=y}=\{x_1,\dots,x_\ell\}$ and $\gd{f(\mathbf x)=y}=0$.
	
	The remaining definitions stay the same.
\end{definition}

\begin{theorem}
	The two following statements are equivalent:
	\begin{enumerate}
		\item nRCR distinguishes $\mathfrak A$ and $\mathfrak B$.
		\item There exists a sentence $\phi\in \mathsf{nfGF}(\mathsf C)$ such that $\mathfrak A\models \phi$ and $\mathfrak B\not\models \phi$.
	\end{enumerate}
\end{theorem}
\begin{proof}
	1. $\Rightarrow$ 2.:
	By definition, $\mathfrak A$ and $\mathfrak B$ are distinguished by nRCR if, and only if, $\mathfrak A'$ and $\mathfrak B'$ are distinguished by RCR.
	Using the result of \cite{scheidt2025ColorRefinement}, we obtain a sentence $\varphi'\in\GFC$ that distinguishes the encoded structures.
	Via a structural induction on the formula, we can now translate $\varphi'$ into a formula $\phi\in \mathsf{nfGF}(\mathsf C)$
	This can be achieved by expanding formulae $R_f(x_1,\dots,x_\ell,y)$ to $f(x_1,\dots,x_\ell)=y$ for function symbols $f\in\sigma$ and letting everything else stay the same.
	
	2. $\Rightarrow$ 1.:
	When considering $\mathsf{nfGF}(\mathsf C)$, one can find that the transformation done at the end of the other direction can be applied in reverse.
	This then leads to a distinguishing sentence in $\GFC$ and with \cite{scheidt2025ColorRefinement} to a distinguishing colouring of the encoded structures, which by definition is a distinguishing colouring for the structures themselves.
\end{proof}

While the above theorem results in a nice characterisation of the naive encoding, the nesting of terms is often very desired when using functions.
However, it can be shown that nesting is too powerful for such a naive encoding.

Consider the two structures $\mathfrak A$ and $\mathfrak B$ of signature $\sigma=\{f/1\}$ which can be seen in \cref{NaiveEncodingCounterexample}.
Formally they are defined as

\begin{alignat*}{2}
	\mathfrak A = (&\{a_1, a_2, &&a_3, a_4, a_5, a_6\} \\ 
	& f^{\mathfrak A} = \{&& \\
	& && a_1 \mapsto a_3,\; a_3 \mapsto a_2,\; a_2 \mapsto a_1, \\
	& && a_4 \mapsto a_5,\; a_5 \mapsto a_6,\; a_6 \mapsto a_4 \\
	&\})
\end{alignat*}
and
\begin{alignat*}{2}
	\mathfrak B = (&\{b_1, b_2, &&b_3, b_4, b_5, b_6\} \\ 
	& f^{\mathfrak A} = \{&& \\
	& && b_1 \mapsto b_3,\; b_3 \mapsto b_5,\; b_5 \mapsto b_6, \\
	& && b_6 \mapsto b_4,\; b_4 \mapsto b_2,\; b_2 \mapsto b_1 \\
	&\})
\end{alignat*}

\begin{figure}[h]
	This will be two nice graphs
	
	\caption{Two $\sigma$-structures $\mathfrak A$ and $\mathfrak B$}
	\label{NaiveEncodingCounterexample}
\end{figure}

Consider the formula $\phi=\exists x.(f(f(f(x)))=x)$ which utilizes term nesting to find a cycle with length three.
It is obvious that $\mathfrak A \models \phi$ and $\mathfrak B\not\models \phi$.
However, when encoding the two structures with the naive method described above, one finds that nRCR cannot distinguish them.
Therefore, term nesting is too powerful for the naive encoding.

A method that allows for the nesting of terms will be described in the following section.




\subsection{Using the bounded transitive closure} 

We use the notation $f^{(m)}(x_1)$ as a shorthand for 
$\underbrace{f(f(\dots f(}_{m\text{-times}}x_1) \dots))$.
\break
Let 
\begin{align*}
\mathcal I(n,m)=\{(k,l,p)\in [n]^3 \quad:\quad & k+p < k+l \leq n \; \land \\
& k+r\cdot l + p = m \text{ for some } r\in \mathbb N\}.
\end{align*}

\begin{definition}
	Let $(i,j)\in \mathbb N^2$ and let $f^\mathfrak A$ be a unary function symbol of a structure $\mathfrak A$. 
	We call $(i,j)$ collision-minimal with respect to $a\in A$, if $i<j$ (especially $i\neq j$),  $f^{(i)}(a)= f^{(j)}(a)$ and for any $(i',j')\in \mathbb N^2$ with the two former properties, either
	\begin{enumerate}
		\item $i < i'$ or
		\item $i=i'$ and $j < j'$.
	\end{enumerate} 
\end{definition}


\begin{lemma}
	Let $\psi(x_1,x_2)\coloneqq f^{(m)}(x_1)=x_2$. 
	Then there exists a formula $\theta(x_1,x_2)$ such that for any $\mathfrak A$ with $\Vert \mathfrak A\Vert=n$ it holds
	$$\mathfrak A \models \psi(\mathbf a) \Longleftrightarrow \mathfrak A \models \theta(\mathbf a)$$ 
	and for any $f^{(m')}(x)$ that appears in $\theta$, $m'\leq n$.
	\label{Simple_fm_to_fk}
\end{lemma}
\begin{proof}
	If $m \leq n$, we let $\theta\coloneqq\psi$ and the claim follows.
	
	Otherwise, we define
	$$\theta(x_1,x_2)\coloneqq \bigvee_{(k,\ell,p)\in \mathcal I(n,m)} \zeta_{(k,\ell,p)}(x_1,x_2)$$
	where
	\begin{align*}
		\zeta_{(k,\ell,p)}(x_1,x_2)\coloneqq & f^{(k+p)}(x_1)=x_2 \land f^{(k)}(x_1)=f^{(k+\ell)}(x_1) \\
		& \land \operatorname{E}^{k,\ell}_{f(x)}(x_1)  \\
		& \land \bigwedge_{\ell'<\ell}f^{(k)}(x_1)\neq f^{(k+\ell')}(x_1)
	\end{align*}
	and
	$$E^{k,\ell}_{f}(t(x_1))=\begin{cases}
		\top & \text{if } k=0 \\
		f^{(k-1)}(t(x_1))\neq f^{(k-1+\ell)}(t(x_1)) & \text{otherwise}.
	\end{cases}$$
	Due to the definition of $\mathcal I(n,m)$ it is obvious that only $f^{(m')}$ with $m'\leq n$ appears.
	
	We now proof the equivalence.
	
	$\Longleftarrow$: Let $\mathfrak A \models \theta(\mathbf a)$. 
	By definition of $\theta$, there are $(k,\ell,p)\in \mathcal I(n,m)$ with $\mathfrak A \models \zeta_{(k,\ell,p)}(a_1,a_2)$.
	In particular $f^{(k)}(a_1)=f^{(k+\ell)}(a_1)$. It follows that
	$$f^{(k)}(a_1)=f^{(k+\ell)}(a_1)=f^{(k+2\ell)}(a_1)=f^{(k+3\ell)}(a_1) = \dots = f^{(k+r\cdot \ell)}(a_1)$$
	for any $r\in \mathbb N$. By using the second equality and the definition of $\mathcal{I}(n,m)$, we get
	$$a_2 =f^{(k+p)}(a_1) = f^{(k+r\cdot \ell + p)}(a_1)=f^{(m)}(a_1).$$
	From this we can deduce $\mathfrak A\models f^{(m)}(a_1)=a_2 \eqqcolon \psi(\mathbf a)$.
	
	$\Longrightarrow$: Let $\mathfrak A\models \psi(a_1,a_2)$. By assumption $m>n$ and by the pigeonhole principle there have to be distinct $i, j$ such that $f^{(i)}(a_1)=f^{(j)}(a_1)$.
	Choose such $i$, $j$ such that they are collision-minimal with respect to $a_1$.
	
	Now choose $k\coloneqq i$, $\ell \coloneqq j-i$ and $p\coloneqq (m-i) \mod (j-i)= (m-i) \mod \ell$.
	Obviously $(k,\ell,p)\in\mathcal I(n,m)$ and what remains to be shown is that $\mathfrak A\models \zeta_{(k,\ell,p)}(a_1,a_2)$.
	
	$\mathfrak A\models f^{(k+p)}(a_1)=a_2$. We use the fact that
	$$a= b \mod c \Leftrightarrow b = r\cdot c +a \text{ for some } r\in \mathbb N.$$
	Then
	$$f^{(k+p)}(a_1)=f^{(i+(m-i)-r\cdot \ell)}(a_1)=f^{(i+r\cdot \ell + m -i - r\cdot \ell)}(a_1)=f^{(m)}(a_1)=a_2.$$
	
	$\mathfrak A\models f^{(k)}(a_1)=f^{(k+\ell)}(a_1)$, because
	$$f^{(k)}(a_1)=f^{(i)}(a_1)=f^{(j)}(a_1)=f^{(j+i-i)}(a_1)=f^{(i+j-i)}(a_1)=f^{(k+\ell)}(a_1).$$
	
	$\mathfrak A\models \operatorname{E}^{k,\ell}_f(a_1)$, because otherwise $f^{(k-1)}(a_1)=f^{(k-1+\ell)}(a)$, but then $(k-1,\ell)$ would contradict the collision-minimality of $(i,j)$.
	
	The same reasoning applies to $\mathfrak A\models \bigwedge_{\ell'<\ell}f^{(k)}(a_1)\neq f^{(k+\ell')}(a_1)$. I.e., otherwise there would be a $(i,j')$ with $j'<j$ which would contradict the collision-minimality.
	
	Thus we have shown that every subformula of the conjunction and therefore the formula is being fulfilled.
\end{proof}

As the following proof, as well as the main argument for the characterisation of $\GFC$, depend on the above proof, a small explanation of the intuition will be stated here.

The indices $(j,k,l)$ represent a path to, the length of and the remaining part of a cycle.
Due to the pigeonhole principle, such a cycle must exist if $m>n$, a picture that describes that argument can be seen in \ref{}.
The argument for first two subformulae of $\zeta$ can be seen in the picture.
$\operatorname{E}^{k,l}_f(x_1)$ and $\bigwedge_{\ell'<\ell}f^{(k)}(x_1)\neq f^{(k+\ell')}(x_1)$ is needed because in the following, it will be required that there is exactly one $(k,\ell,p)$ with the above properties.

\begin{lemma}
	Let $\psi(x_1,x_2)\coloneqq t(x_1)=x_2$ be an atomic formula. 
	Then there exists a $\theta_{t(x)}(x_1,x_2)$ such that for any structure (of a fitting signature) $\mathfrak A$ with $\Vert \mathfrak A \Vert = n$ it holds
	$$\mathfrak A \models \psi(\mathbf a) \Longleftrightarrow \mathfrak A \models \vartheta_{t(x)}(\mathbf a).$$ 
	Furthermore, $\theta_{t(x)}(x_2,x_2)$ is of the form $\theta_{t(x)}(x_1,x_2)=\bigvee \Phi(x_1,x_2)$ where all $\phi(x_1,x_2)\in\Phi(x_1,x_2)$ are of the form
	$$\phi(x_1,x_2)\coloneqq \Delta(x_1,x_2) \land \bigwedge \Psi(x_1)$$ 
	where $\Delta(x_1,x_2)\coloneqq t'(x_1)=x_2$ for some term $t'(x_1)$, and for every $f^m(s(x))$ that appears in $\theta_{t(x)}$, $m\leq n$.
\end{lemma}
\begin{proof}
	We proof this via an induction on the term $t(x_1)$.
	
	\textbf{Base cases: } 
	If $t(x_1)\coloneqq x_1$, we define $\theta_{t(x)}(x_1,x_2)=\bigvee \{\Delta(x_1,x_2) \land \bigwedge \Psi\}$ with $\Delta(x_1,x_2) \coloneqq x_1=x_2$ and $\Psi=\{\top\}$.
	Thus obviously $\theta_{t(x)}(x_1,x_2) \equiv x_1 = x_2$.
	
	If $t(x_1)\coloneqq f^{(m)}(x_1)$ for a unary function symbol $f$ and $m\in \mathbb N$, we use the formula constructed in the proof of \cref{Simple_fm_to_fk}.
	It can easily be verified that it is in the correct form.
	
	\textbf{Inductive step: }
	Assume that $t(x_1)\coloneqq g^m(s(x_1))$ for a function symbol $g$, $m\in\mathbb N$ and term $s$.
	By induction hypothesis, we have a formula $\theta_{s(x)}(y_1,y_2)=\bigvee \Phi_s$ in the above defined form with $\mathfrak A,\mathbf{a} \models s(y_1)=y_2 \Leftrightarrow \mathfrak A,\mathbf{a}\models \theta_{s(x)}(y_1,y_2)$.
	
	If $m\leq n$, we set 
	$$\theta_{t(x_1)}(x_1,x_2)=\bigvee \Phi'(x_1,x_2),$$
	where $\Phi'(x_1,x_2)\coloneqq\{g^{(m)}(t'(y_1/x_1))=x_2 \land \bigwedge \Psi(y_1/x_1) : t'(y_1)=y_2 \land \bigwedge \Psi(y_1)\in \Phi_s\}$.
	
	Now assume $m>n$.
	
	Then we set
	$$\theta_{t(x_1)}(x_1,x_2)=\bigvee_{(k,\ell,p)\in \mathcal I(n,m)} \bigvee \Phi'_{(k,\ell,p)}(x_1,x_2),$$
	where 
	\begin{align*}
		\Phi'_{(k,\ell,p)}\coloneqq \{g^{(k+p)}(t'(y_1/x_1))=x_2 &\land g^{(k)}(t'(y_1/x_1))=g^{(k+l)}(t'(y_1/x_1)) \\
		& \land \operatorname{E}^{k,l}_g(t'(y_1/x_1)) \land \bigwedge_{\ell'<\ell} g^{(k)}(t'(y_1/x_1))\neq g^{(k+\ell')}(t'(y_1/x_1)) \\
		& \land \Psi(y_1/x_1) : t'(y_1)=y_2 \land \bigwedge \Psi(y_1)\in \Phi_s\}
	\end{align*}
	
	By using the above definitions, we get $\mathfrak A,\mathbf a\models s(y_1)=y_2 \Leftrightarrow \mathfrak A,\mathbf a\models \phi_s(y_1,y_2)\coloneqq t'(y_1)=y_2 \land \bigwedge \Psi(y_1))$ for some $\phi_s\in\Phi_s$.
	Therefore
	\begin{equation}
		\mathfrak A, \mathbf a \models s(y_1)=y_2 \Longleftrightarrow \mathfrak A,\mathbf a \models t'(y_1)=y_2 \land \bigwedge \Psi(y_1).
		\label{Equivalence_s_and_tPsi}
	\end{equation}
	
	We now proof that 
	$$\mathfrak A, \mathbf a\models t(x_1)=x_2 \Longleftrightarrow \mathfrak A,\mathbf a\models \theta_{t(x_1)}(x_1,x_2).$$
	
	$\Longleftarrow$: 
	Let $\mathfrak A, \mathbf a \models \theta_{t(x_1)}$.
	If $m\leq n$, then there is some $\psi(x_1,x_2)\coloneqq g^{(m)}(t'(y_1/x_1))=x_2 \land \bigwedge \Psi(y_1/x_1)$ such that $\mathfrak A,a_1,a_2\models \psi(x_1,x_2)$.
	We then get
	\begin{align*}
		\mathfrak A,a_1,a_2 \models & g^{(m)}(t'(y_1/x_1))=x_2 \land \bigwedge \Psi(y_1/x_1) \\
		\Leftrightarrow \mathfrak A,a_1,a_2,a_3 \models & g^{(m)}(x_3)=x_2 \land \bigwedge \Psi(y_1/x_1) \land t'(y_1/x_1)=x_3 \text{ for any } a_3\in A \\
		\overset{\cref{Equivalence_s_and_tPsi}}{\Leftrightarrow} \mathfrak A,a_1,a_2,a_3\models & g^{(m)}(x_3)=x_2 \land s(x_1)=x_3 \text{ for any } a_3\in A \\
		\Leftrightarrow \mathfrak A,a_1,a_2 \models & g^{(m)}(s(x_1))=x_2.
	\end{align*}
	This was to be shown.
	
	Now let $m>n$.
	Now there is a 
	\begin{align*}
		\psi(x_1,x_2)\coloneqq g^{(k+p)}(t'(y_1/x_1))=x_2 &\land g^{(k)}(t'(y_1/x_1))=g^{(k+l)}(t'(y_1/x_1)) \\
		& \land \operatorname{E}^{k,l}_g(t'(y_1/x_1)) \land \bigwedge_{\ell'<\ell} g^{(k)}(t'(y_1/x_1))\neq g^{(k+\ell')}(t'(y_1/x_1)) \\
		& \land \Psi(y_1/x_1)
	\end{align*}
	for some $(k,\ell,p)\in\mathcal I(n,m)$ with $\mathfrak A,a_1,a_2\models \psi(x_1,x_2)$.
	And now
	\begin{align*}
		\mathfrak A,a_1,a_2 \models & \psi(x_1,x_2) \\
		\Leftrightarrow A,a_1,a_2,a_3 \models &g^{(k+p)}(x_3)=x_2 \land g^{(k)}(x_3))=g^{(k+l)}(x_3) \\
			& \land \operatorname{E}^{k,l}_g(x_3) \land \bigwedge_{\ell'<\ell} g^{(k)}(x_3))\neq g^{(k+\ell')}(x_3)) \\
			& \land \Psi(y_1/x_1) \land t'(y_1/x_1)=x_3 \text{ for any } a_3\in A \\
		\overset{\cref{Simple_fm_to_fk}}{\Leftrightarrow} \mathfrak A, a_1,a_2,a_3 \models & g^{(m)}(x_3)=x_2 \land t'(y_1/x_1)=x_3 \land \Psi(y_1/x_1) \text{ for any } a_3 \in A \\
		\overset{\cref{Equivalence_s_and_tPsi}}{\Leftrightarrow} \mathfrak A,a_1,a_2,a_3 \models & g^{(m)}(x_3)=x_2\land s(x_1)=x_3 \text{ for any } a_3\in A \\
		\Leftrightarrow \mathfrak A,a_1,a_2 \models & g^{(m)}(s(x_1))=x_2.
	\end{align*}
	This concludes the direction.
	
	The other direction should be analogous, as for any set of formulae $\Phi$ and formulae $\psi\in \Phi$ it holds: $\mathfrak A \models \psi \Leftrightarrow \mathfrak A\models \bigvee \Phi$.
\end{proof}

\section {Relational Colour Refinement for symmetric structures}

\lipsum[3-4]

\section{Conclusion}

\lipsum[2-3]


\clearpage

\bibliographystyle{plainurl}
\bibliography{references.bib}





\end{document}





