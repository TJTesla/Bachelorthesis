%%%%%%%%%%%%%%%%%%%%%%%%%%%%%%%%%%%%%%%%%%%%%%%%%%%%%%%%%%%%%%%%%%%%%%%%%%%%%%%
% i7 Seminar Report Template
% Version November 8, 2023

\documentclass[a4paper,11pt,DIV=15]{scrartcl} % Do not edit this line.


%%%%%%%%%%%%%%%%%%%%%%%%%%%%%%%%%%%%%%%%%%%%%%%%%%%%%%%%%%%%%%%%%%%%%%%%%%%%%%%
% Preamble

% Page Geometry, Typography and Encoding
\usepackage[utf8]{inputenc}
\usepackage[T1]{fontenc}
\usepackage{microtype}
\usepackage{lmodern}
\renewcommand{\phi}{\varphi}
\renewcommand{\epsilon}{\varepsilon}
% \renewcommand{theta}{\vartheta} % if you want

% Math packages
\usepackage{amsmath}
\usepackage{amssymb}
\usepackage{amsthm}
\usepackage{mathtools}

% Floats
\usepackage{float}
\usepackage{booktabs}
\usepackage{tikz}
\usetikzlibrary{positioning,arrows.meta}

% Colors
\usepackage{xcolor} %already loaded by tikz, but here for completeness
% RWTH colors
% blue violet purple carmine red magenta orange yellow grass cyan gold silver
\definecolor{rwth-blue}{cmyk}{1,.5,0,0}\colorlet{rwth-lblue}{rwth-blue!50}\colorlet{rwth-llblue}{rwth-blue!25}
\definecolor{rwth-violet}{cmyk}{.6,.6,0,0}\colorlet{rwth-lviolet}{rwth-violet!50}\colorlet{rwth-llviolet}{rwth-violet!25}
\definecolor{rwth-purple}{cmyk}{.7,1,.35,.15}\colorlet{rwth-lpurple}{rwth-purple!50}\colorlet{rwth-llpurple}{rwth-purple!25}
\definecolor{rwth-carmine}{cmyk}{.25,1,.7,.2}\colorlet{rwth-lcarmine}{rwth-carmine!50}\colorlet{rwth-llcarmine}{rwth-carmine!25}
\definecolor{rwth-red}{cmyk}{.15,1,1,0}\colorlet{rwth-lred}{rwth-red!50}\colorlet{rwth-llred}{rwth-red!25}
\definecolor{rwth-magenta}{cmyk}{0,1,.25,0}\colorlet{rwth-lmagenta}{rwth-magenta!50}\colorlet{rwth-llmagenta}{rwth-magenta!25}
\definecolor{rwth-orange}{cmyk}{0,.4,1,0}\colorlet{rwth-lorange}{rwth-orange!50}\colorlet{rwth-llorange}{rwth-orange!25}
\definecolor{rwth-yellow}{cmyk}{0,0,1,0}\colorlet{rwth-lyellow}{rwth-yellow!50}\colorlet{rwth-llyellow}{rwth-yellow!25}
\definecolor{rwth-grass}{cmyk}{.35,0,1,0}\colorlet{rwth-lgrass}{rwth-grass!50}\colorlet{rwth-llgrass}{rwth-grass!25}
\definecolor{rwth-green}{cmyk}{.7,0,1,0}\colorlet{rwth-lgreen}{rwth-green!50}\colorlet{rwth-llgreen}{rwth-green!25}
\definecolor{rwth-cyan}{cmyk}{1,0,.4,0}\colorlet{rwth-lcyan}{rwth-cyan!50}\colorlet{rwth-llcyan}{rwth-cyan!25}
\definecolor{rwth-teal}{cmyk}{1,.3,.5,.3}\colorlet{rwth-lteal}{rwth-teal!50}\colorlet{rwth-llteal}{rwth-teal!25}
\definecolor{rwth-gold}{cmyk}{.35,.46,.7,.35}
\definecolor{rwth-silver}{cmyk}{.39,.31,.32,.14}

% Hyperlinks and Cross-References
\usepackage{hyperref}
\usepackage[capitalise,noabbrev]{cleveref}
\hypersetup{%
	pdftoolbar=false,
	pdfmenubar=false,
	colorlinks,
	%pdfborderstyle={/S/U/W 1.25},
	urlcolor={rwth-magenta},
	linkcolor={rwth-red},
	citecolor={rwth-green}
}

\theoremstyle{plain}
\newtheorem{theorem}{Theorem}
\newtheorem{proposition}[theorem]{Proposition}
\newtheorem{lemma}[theorem]{Lemma}
\newtheorem{corollary}[theorem]{Corollary}
\newtheorem{conjecture}[theorem]{Conjecture}
\newtheorem{claim}[theorem]{Claim}
\theoremstyle{definition}
\newtheorem{definition}[theorem]{Definition}
\newtheorem{remark}[theorem]{Remark}



% Misc packages
\usepackage{lipsum}


% Custom commands
\newcommand{\GFC}{\mathsf{GF}(\mathsf{C})}
\newcommand{\free}[1]{\operatorname{free}(#1)}
\newcommand{\gd}[1]{\operatorname{gd}(#1)}


%%%%%%%%%%%%%%%%%%%%%%%%%%%%%%%%%%%%%%%%%%%%%%%%%%%%%%%%%%%%%%%%%%%%%%%%%%%%%%%
% Document


\begin{document}

%TODO Insert topic of seminar, e.g. Theoretical Topics in Data Science or Complexity Theory
\subtitle{Bachelor's Thesis}
\date{August 28, 2025}
\publishers{RWTH Aachen University}	% Do not edit this line.

%TODO Change this to your report title.
\title{Relational Colour Refinement for Non-Relational Signatures}

%TODO Change this to your name.
\author{Theodor Teslia}

\maketitle


%TODO Provide a short abstract for your report.
\begin{abstract}
	\lipsum[1-2]
\end{abstract}

\thispagestyle{empty}

\clearpage

%TODO The content of your report goes below.


\section{Introduction}

\lipsum[2-3]

\section{Relational Colour Refinement}

\lipsum[4-5]

\clearpage

\section{Relational Colour Refinement for structures with functions}

\subsection{Naive Encoding of functions}

A simple way to apply relational colour refinement to non-relational structures is, to encode the functions in the signature as a relation.
Formally we transform a signature $\sigma$ that includes function symbols to a new signature $\sigma'$: 
For every relation symbol $R\in \sigma$, we introduce a relation symbol $R\in \sigma'$ with the same arity and for every function symbol $f\in\sigma$ with arity $k$, we introduce a relational symbol $R_f\in\sigma'$ of arity $k+1$.

Semantically, a structure $\mathfrak A$ of signature $\sigma$ can then be encoded as a structure $\mathfrak A'$ of signature $\sigma'$ and with the same universe as $\mathfrak A$. 
For every relational symbol $R\in\sigma$ we set $R^{\mathfrak A'}\coloneqq R^{\mathfrak A}$ and for every function symbol $f\in\sigma$ of arity $k$ there exists a relation symbol $R_f\in\sigma'$ and we set $R_f^{\mathfrak A}\coloneqq \{(\mathbf x, y) : f^{\mathfrak A}(\mathbf x)=y\}$ where $\mathbf x$ is a tuple of arity $k$.

This procedure encodes a non-relational structure as a relational one, on which Relational Colour Refinement can now be performed.
As such we say, that the Naive Relational Colour Refinement (nRCR) distinguishes two structures $\mathfrak A$ and $\mathfrak B$ if, and only if, RCR distinguishes their naive encodings $\mathfrak A'$ and $\mathfrak B'$.
However, this results in a very weak logical characterisation, that does not allow nesting of terms, namely the nesting-free-fragment of $\GFC$.

\begin{definition}[$\mathsf{nfGF}(\mathsf C)$]
	Consider the definition of $\GFC$ given in \ref{}.
	We obtain the nesting-free fragment, by allowing $f(\mathbf x)=y$ as a further atomic formula.
	Concretely, the only allowed atomic formulae are of the form $R(x_1,\dots,x_\ell)$, $x=y$ and $f(x_1,\dots,x_\ell)=y$, where $f$ has arity $\ell$, $\free{f(x_1,\dots,x_\ell)=y}=\{x_1,\dots,x_\ell\}$ and $\gd{f(\mathbf x)=y}=0$.
	
	The remaining definitions stay the same.
\end{definition}

\begin{theorem}
	The two following statements are equivalent:
	\begin{enumerate}
		\item nRCR distinguishes $\mathfrak A$ and $\mathfrak B$.
		\item There exists a sentence $\phi\in \mathsf{nfGF}(\mathsf C)$ such that $\mathfrak A\models \phi$ and $\mathfrak B\not\models \phi$.
	\end{enumerate}
\end{theorem}
\begin{proof}
	1. $\Rightarrow$ 2.:
	By definition, $\mathfrak A$ and $\mathfrak B$ are distinguished by nRCR if, and only if, $\mathfrak A'$ and $\mathfrak B'$ are distinguished by RCR.
	Using the result of \cite{scheidt2025ColorRefinement}, we obtain a sentence $\varphi'\in\GFC$ that distinguishes the encoded structures.
	Via a structural induction on the formula, we can now translate $\varphi'$ into a formula $\phi\in \mathsf{nfGF}(\mathsf C)$
	This can be achieved by expanding formulae $R_f(x_1,\dots,x_\ell,y)$ to $f(x_1,\dots,x_\ell)=y$ for function symbols $f\in\sigma$ and letting everything else stay the same.
	
	2. $\Rightarrow$ 1.:
	When considering $\mathsf{nfGF}(\mathsf C)$, one can find that the transformation done at the end of the other direction can be applied in reverse.
	This then leads to a distinguishing sentence in $\GFC$ and with \cite{scheidt2025ColorRefinement} to a distinguishing colouring of the encoded structures, which by definition is a distinguishing colouring for the structures themselves.
\end{proof}

While the above theorem results in a nice characterisation of the naive encoding, the nesting of terms is often very desired when using functions.
However, it can be shown that nesting is too powerful for such a naive encoding.

Consider the two structures $\mathfrak A$ and $\mathfrak B$ of signature $\sigma=\{f/1\}$ which can be seen in \cref{NaiveEncodingCounterexample}.
Formally they are defined as

\begin{alignat*}{2}
	\mathfrak A = (&\{a_1, a_2, &&a_3, a_4, a_5, a_6\} \\ 
	& f^{\mathfrak A} = \{&& \\
	& && a_1 \mapsto a_3,\; a_3 \mapsto a_2,\; a_2 \mapsto a_1, \\
	& && a_4 \mapsto a_5,\; a_5 \mapsto a_6,\; a_6 \mapsto a_4 \\
	&\})
\end{alignat*}
and
\begin{alignat*}{2}
	\mathfrak B = (&\{b_1, b_2, &&b_3, b_4, b_5, b_6\} \\ 
	& f^{\mathfrak A} = \{&& \\
	& && b_1 \mapsto b_3,\; b_3 \mapsto b_5,\; b_5 \mapsto b_6, \\
	& && b_6 \mapsto b_4,\; b_4 \mapsto b_2,\; b_2 \mapsto b_1 \\
	&\})
\end{alignat*}

\begin{figure}[h]
	This will be two nice graphs
	
	\caption{Two $\sigma$-structures $\mathfrak A$ and $\mathfrak B$}
	\label{NaiveEncodingCounterexample}
\end{figure}

Consider the formula $\phi=\exists x.(f(f(f(x)))=x)$ which utilizes term nesting to find a cycle with length three.
It is obvious that $\mathfrak A \models \phi$ and $\mathfrak B\not\models \phi$.
However, when encoding the two structures with the naive method described above, one finds that nRCR cannot distinguish them.
Therefore, term nesting is too powerful for the naive encoding.

A method that allows for the nesting of terms will be described in the following section.

\subsection{Using the bounded transitive closure} 

\begin{theorem}
	Let $\psi(x_1,x_2)\coloneqq f^m(x_1) = x_2$.
\end{theorem}

\section {Relational Colour Refinement for symmetric structures}

\lipsum[3-4]

\section{Conclusion}

\lipsum[2-3]


\clearpage

\bibliographystyle{plainurl}
\bibliography{references.bib}





\end{document}





