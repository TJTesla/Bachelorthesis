%%%%%%%%%%%%%%%%%%%%%%%%%%%%%%%%%%%%%%%%%%%%%%%%%%%%%%%%%%%%%%%%%%%%%%%%%%%%%%%
% i7 Seminar Report Template
% Version November 8, 2023

\documentclass[a4paper,11pt,DIV=15]{scrartcl} % Do not edit this line.


%%%%%%%%%%%%%%%%%%%%%%%%%%%%%%%%%%%%%%%%%%%%%%%%%%%%%%%%%%%%%%%%%%%%%%%%%%%%%%%
% Preamble

% Page Geometry, Typography and Encoding
\usepackage[utf8]{inputenc}
\usepackage[T1]{fontenc}
\usepackage{microtype}
\usepackage{lmodern}
\renewcommand{\phi}{\varphi}
\renewcommand{\epsilon}{\varepsilon}
% \renewcommand{theta}{\vartheta} % if you want

% Math packages
\usepackage{amsmath}
\usepackage{amssymb}
\usepackage{amsthm}
\usepackage{mathtools}

% Floats
\usepackage{float}
\usepackage{booktabs}
\usepackage{tikz}
\usetikzlibrary{positioning,arrows.meta}

% Colors
\usepackage{xcolor} %already loaded by tikz, but here for completeness
% RWTH colors
% blue violet purple carmine red magenta orange yellow grass cyan gold silver
\definecolor{rwth-blue}{cmyk}{1,.5,0,0}\colorlet{rwth-lblue}{rwth-blue!50}\colorlet{rwth-llblue}{rwth-blue!25}
\definecolor{rwth-violet}{cmyk}{.6,.6,0,0}\colorlet{rwth-lviolet}{rwth-violet!50}\colorlet{rwth-llviolet}{rwth-violet!25}
\definecolor{rwth-purple}{cmyk}{.7,1,.35,.15}\colorlet{rwth-lpurple}{rwth-purple!50}\colorlet{rwth-llpurple}{rwth-purple!25}
\definecolor{rwth-carmine}{cmyk}{.25,1,.7,.2}\colorlet{rwth-lcarmine}{rwth-carmine!50}\colorlet{rwth-llcarmine}{rwth-carmine!25}
\definecolor{rwth-red}{cmyk}{.15,1,1,0}\colorlet{rwth-lred}{rwth-red!50}\colorlet{rwth-llred}{rwth-red!25}
\definecolor{rwth-magenta}{cmyk}{0,1,.25,0}\colorlet{rwth-lmagenta}{rwth-magenta!50}\colorlet{rwth-llmagenta}{rwth-magenta!25}
\definecolor{rwth-orange}{cmyk}{0,.4,1,0}\colorlet{rwth-lorange}{rwth-orange!50}\colorlet{rwth-llorange}{rwth-orange!25}
\definecolor{rwth-yellow}{cmyk}{0,0,1,0}\colorlet{rwth-lyellow}{rwth-yellow!50}\colorlet{rwth-llyellow}{rwth-yellow!25}
\definecolor{rwth-grass}{cmyk}{.35,0,1,0}\colorlet{rwth-lgrass}{rwth-grass!50}\colorlet{rwth-llgrass}{rwth-grass!25}
\definecolor{rwth-green}{cmyk}{.7,0,1,0}\colorlet{rwth-lgreen}{rwth-green!50}\colorlet{rwth-llgreen}{rwth-green!25}
\definecolor{rwth-cyan}{cmyk}{1,0,.4,0}\colorlet{rwth-lcyan}{rwth-cyan!50}\colorlet{rwth-llcyan}{rwth-cyan!25}
\definecolor{rwth-teal}{cmyk}{1,.3,.5,.3}\colorlet{rwth-lteal}{rwth-teal!50}\colorlet{rwth-llteal}{rwth-teal!25}
\definecolor{rwth-gold}{cmyk}{.35,.46,.7,.35}
\definecolor{rwth-silver}{cmyk}{.39,.31,.32,.14}

% Hyperlinks and Cross-References
\usepackage{hyperref}
\usepackage[capitalise,noabbrev]{cleveref}
\hypersetup{%
	pdftoolbar=false,
	pdfmenubar=false,
	colorlinks,
	%pdfborderstyle={/S/U/W 1.25},
	urlcolor={rwth-magenta},
	linkcolor={rwth-red},
	citecolor={rwth-green}
}

\theoremstyle{plain}
\newtheorem{theorem}{Theorem}
\newtheorem{proposition}[theorem]{Proposition}
\newtheorem{lemma}[theorem]{Lemma}
\newtheorem{corollary}[theorem]{Corollary}
\newtheorem{conjecture}[theorem]{Conjecture}
\newtheorem{claim}[theorem]{Claim}
\theoremstyle{definition}
\newtheorem{definition}[theorem]{Definition}
\newtheorem{remark}[theorem]{Remark}



% Misc packages
\usepackage{lipsum}



%%%%%%%%%%%%%%%%%%%%%%%%%%%%%%%%%%%%%%%%%%%%%%%%%%%%%%%%%%%%%%%%%%%%%%%%%%%%%%%
% Document


\begin{document}

%TODO Insert topic of seminar, e.g. Theoretical Topics in Data Science or Complexity Theory
\subtitle{}
\date{\today}
\publishers{RWTH Aachen University}	% Do not edit this line.

%TODO Change this to your report title.
\title{Thesis Concept: Relational Color Refinement}

%TODO Change this to your name.
\author{Theodor Teslia}

\maketitle

\section{Background}

Color Refinement is a well-understood algorithm that can be applied in a lot of situations.
The original application of the procedure is graph isomorphism testing, as it is possible to prove that two graphs are not isomorphic (however proving isomorphy is not possible).
However, there are also other fields, for example Linear Programming, where the algorithm can be used as a pre-processing step to find a smaller, but equivalent, problem or Machine Learning, where color refinement can accelerate computations for kernel methods. \cite{grohe2021color}
\\ \\
The wide applicability and very simple nature of color refinement poses the question, whether an analogous algorithm exists for general relational structures.
Such a relational color refinement (RCR) has been found by Schweikardt and Scheidt \cite{scheidt2024color} and will be further described in the following.
\\ \\
The key idea of RCR is to color tuples of elements that appear in some relation instead of the specific elements themselves. 
The coloring uses the atomic type $\mathsf{atp}(\mathbf{a})=\{R\in\sigma : \mathbf{a}\in R\}$ (the set of relations that include $\mathbf{a}$) as well as the similarity type $\mathsf{stp}(\mathbf{a},\mathbf{b})=\{(i,j)\in[k]\times[\ell] : a_i=b_j\}$ (the set of pairs of indices for equal elements) where $\mathbf{a}$ and $\mathbf{b}$ are of arity $k$ and $\ell$ respectively.
This algorithm is shown to be equivalent to color refinement on a graph that is constructed from the relational structure by using the tuples as vertices and the shared types as edges.
\\ \\
An interesting aspect of color refinement is the equivalence to counting the number of homomorphisms between trees and the graphs \cite{dvovrak2010recognizing}. A similar method can be found for relational color refinement.
Needed for this is the generalization of trees (i.e. acyclic, connected graphs) to acyclic relational structures.
For this we define the join-tree $J$, which represents a structure $\mathfrak C$ as a tree, where the nodes are the tuples of elements $\mathbf c$ that appear in some relation $R^\mathfrak C$. Further the edges are defined, such that for every element $v\in C$, the set $\{\mathbf{c}\in \mathbf{C} : v\in\operatorname{set}(\mathbf c)\}$ induces a connected subgraph (which is also a tree).
Then an acyclic relational structure is a structure for which a join tree exists.
With this we can obtain the following equivalence:
\begin{center}
$\sigma$-structures $\mathfrak{A}$ and $\mathfrak{B}$ are distinguished by Relational Color Refinement \\ $\Leftrightarrow$ \\ There exists an acyclic $\sigma$-structure $\mathfrak{C}$ s.t $\hom(\mathfrak C, \mathfrak A)\neq\hom(\mathfrak C,\mathfrak B)$.
\end{center}
where $\hom(\mathfrak C,\mathfrak A)$ is the number of homomorphisms from $\mathfrak C$ to $\mathfrak A$.
\\ \\
Another way to characterize graphs that are distinguished by color refinement is logically.
As described in \cite{cai1992optimal}, two graphs are distinguished by color refinement iff. there exists a sentence in first-order logic with counting quantifiers and up to $2$ variables such that one graph fulfills the sentence and the other does not.
Again a similar result has been achieved for relational color refinement.
By taking formulae of the guarded fragment of first-order counting logic the equivalence holds for general relational structures.
The guarded fragment can informally be described as those formulae which have a guard for every quantifier, i.e. quantifiers only appear in the form $\exists^{\geq n}\mathbf v.(\Delta \land \psi)$ where the free variables in $\Delta$ are a superset of the free variables of $\psi$. $\Delta$ is then called a guard for $\psi$.

Similarly to the analogous result for graphs the proof uses a game between Spoiler and Duplicator to prove the above results.
\\ \\
By combining the equivalences, we obtain the end-result of \cite{scheidt2024color}:

\begin{theorem}[Scheidt, Schweikardt \cite{scheidt2024color}]
	Let $\sigma$ be a (relational) signature and let $\mathfrak A$ and $\mathfrak B$ be $\sigma$-structures. The following statements are equivalent:
	\begin{enumerate}
		\item Relational Color Refinement distinguishes $\mathfrak A$ and $\mathfrak B$.
		\item There is an acyclic $\sigma$-structure $\mathfrak C$ such that $\hom(\mathfrak C,\mathfrak A)\neq\hom(\mathfrak C,\mathfrak B)$.
		\item There is a sentence $\varphi\in \mathsf{GF(C)}$ such that $\mathfrak A\models\varphi$ and $\mathfrak B\not\models\varphi$.
		\item Spoiler has a winning strategy for the Guarded-Game on $\mathfrak A,\mathfrak B$.
	\end{enumerate}
	\label{RCRMainThm}
\end{theorem}

\section{Objectives}
\label{Sec::Objectives}

The objectives and aims of the thesis are to explore the robustness of the results of \cite{scheidt2024color}. Primarily the applicability and adaptability to other classes of structures will be examined.

Those include
\begin{itemize}
	\item Structures with functions, in particular constants / $0$-ary functions
	\item Symmetric structures
	\item Ordered structures
	\item Other classes
\end{itemize}
on which I will be elaborating in the following.

\subsection{Structures with Functions}

In the considerations for \cref{RCRMainThm}, there do not appear function-symbols, since only relational structures are considered.
However, as the generalization from graphs to relational structures is possible, it poses the question whether a further widening to arbitrary signatures is achievable.
For this, constant symbols, unary functions, and functions of arbitrary arity could be included in the signature.

A first consideration in the thesis then needs to be, whether the given definitions still make sense, i.e. can $\mathsf{GF(C)}$ also work with functions, do functions impact the acyclic property of structures?
Further the definition of RCR needs to be investigated and adapted to also work with function symbols.

As they are considerably simpler, those considerations will first be done for constants, i.e. $0$-ary functions and then increased to functions of higher arity.
\label{StructWFunc}

\subsection{Symmetric Structures}

A symmetric structure is a relational structure $\mathfrak A$, where for all relations $R^\mathfrak A$ of arity $k$ it holds $R^\mathfrak A \subseteq \binom{A}{k}$, i.e. the order of elements inside a tuple does not matter.

As those structures are relational, \cref{RCRMainThm} obviously applies to them.
However this trivial implication does not apply if we also restrict the other sides of the equivalence.
In particular, does it still hold, if instead of acyclic structures we only use symmetric acyclic structures?
Or can we restrict parts of the distinguishing sentence to be of a certain form?

A possible approach could be inspired by \cite{scheidt2023counting}, since symmetric structures and hypergraphs seem to be similar in some form.
\label{SymmStruct}

\subsection{Ordered Structures}

Given a relational structure where we require the structure to be ordered by a partial / total ordering, what role does the order play on the coloring?
Furthermore the questions in \cref{SymmStruct} can be investigated.

\subsection{Other classes}

Additionally, the investigation of other classes of structures is possible, but will not be focused on as intensely as the above classes.
These could, for example, be algebraic structures, which are a subclass of structures with functions, or relational structures with (countably) infinite signatures, which would be an expansion on relational structures.
This list is obviously non-exhaustive.

\subsection{Further Examinations}

Another possible question can be stated in relation to the results and methods of \cite{dell2018lov}.
The term ``fractionally isomorphic'' is defined and is shown to be equivalent to the classical color refinement results.
Can such a concept be widened, to also be equivalent to the relational color refinement?
If not, why?

\section{Structure}

First I will begin with an introduction to color refinement on graphs and then define the relational color refinement due to \cite{scheidt2024color}.
Further the existing work will be put into context by naming related works.

After explaining the underlying algorithms, logics and structures, such that the contents of \cite{scheidt2024color} are understandable, I will move onto the questions and problems stated in \cref{Sec::Objectives}.
The main focus will lie on structures with functions and symmetric structures.
However while the other classes will be investigated too, the main work should happen in the chapters for the two former classes.

At the end, I will conclude my results, put them into context, point out possible problems and state approaches for further work.


\section{Related Works}

Color refinement is a well-known algorithm that allows for isomorphism testing and performance optimizations in various fields, such as linear programming or machine learning \cite{grohe2021color}.
There have been multiple considerations of the homomorphism indistinguishability defined by color refinement (e.g. \cite{fluck2023going}, \cite{manvcinska2020quantum}).
Furthermore, the $k$-dimensional Weisfeiler-Leman Algorithm extends the simple procedure and allows for stricter relationships between graphs \cite{kiefer2020power}.
However, the established results are constrained to (subclasses of) graphs (or colored multigraphs, cf. \cite{dell2018lov}).
An extension has been achieved by Schweickardt and Scheidt, by showing an analogous result for hypergraphs (graphs that admit edges between any number of vertices) in \cite{scheidt2023counting} and has further been generalized to structures of arbitrary relational signatures \cite{scheidt2024color}.
As color refinement on graphs has shown itself to be very applicable in a lot of situations, a similar usefulness for relational color refinement seems probable.


\bibliographystyle{plainurl}
\bibliography{references.bib}





\end{document}





